\newcommand\nuclear{\omding\char195\xspace}
\newcommand\goodskull{\omding\char194\xspace}

\newcommand{\Co}{\ensuremath{\mathbb{C}}}

\newcommand{\smallsum}{\textstyle{\sum\limits_i}}
\newcommand{\smallsumi}[1]{\textstyle{\sum\limits_{#1}}}

\newcommand{\emptydiag}{\,\tikz{\node[style=empty diagram] (x) {};}\,}
\newcommand{\scalar}[1]{\,\tikz{\node[style=scalar] (x) {$#1$};}\,}
\newcommand{\dscalar}[1]{\,\tikz{\node[style=scalar, doubled] (x) {$#1$};}\,}
\newcommand{\boxmap}[1]{\,\tikz{\node[style=small box] (x) {$#1$};\draw(0,-1)--(x)--(0,1);}\,}
\newcommand{\tboxmap}[3]{\,\begin{tikzpicture}
  \begin{pgfonlayer}{nodelayer}
    \node [style=small box] (0) at (0, 0) {$#1$};
    \node [style=none] (1) at (0, -0.5) {};
    \node [style=none] (2) at (0, -1.25) {};
    \node [style=none] (3) at (0, 1.25) {};
    \node [style=none] (4) at (0, 0.5) {};
    \node [style=right label] (5) at (0.25, -1) {$#2$};
    \node [style=right label] (6) at (0.25, 1) {$#3$};
  \end{pgfonlayer}
  \begin{pgfonlayer}{edgelayer}
    \draw (2.center) to (1.center);
    \draw (4.center) to (3.center);
  \end{pgfonlayer}
\end{tikzpicture}}
\newcommand{\boxstate}[1]{\,\tikz{\node[style=small box] (x) {$#1$};\draw(x)--(0,1);}\,}
\newcommand{\boxeffect}[1]{\,\tikz{\node[style=small box] (x) {$#1$};\draw(0,-1)--(x);}\,}
\newcommand{\boxmapTWOtoONE}[1]{\,\tikz{\node[style=small box] (x) {$#1$};\draw(-0.8,-1)--(x)--(0,1);\draw (0.8,-1)--(x);}\,}
\newcommand{\boxmapONEtoTWO}[1]{\,\tikz{\node[style=small box] (x) {$#1$};\draw(0,-1)--(x)--(-1,1);\draw (x)--(1,1);}\,}

\newcommand{\doubleop}{\ensuremath{\textsf{double}}\xspace}

\newcommand{\kscalar}[1]{\,\tikz{\node[style=kscalar] (x) {$#1$};}\,}
\newcommand{\kscalarconj}[1]{\,\tikz{\node[style=kscalarconj] (x) {$#1$};}\,}

\newcommand{\dmap}[1]{\,\tikz{\node[style=dmap] (x) {$#1$};\draw[boldedge] (0,-1.3)--(x)--(0,1.3);}\,}
\newcommand{\dmapdag}[1]{\,\tikz{\node[style=dmapdag] (x) {$#1$};\draw[boldedge] (0,-1.3)--(x)--(0,1.3);}\,}

\newcommand{\map}[1]{\,\tikz{\node[style=map] (x) {$#1$};\draw(0,-1.3)--(x)--(0,1.3);}\,}

\newcommand{\mapdag}[1]{\,\tikz{\node[style=mapdag] (x) {$#1$};\draw(0,-1.3)--(x)--(0,1.3);}\,}

\newcommand{\maptrans}[1]{\,\tikz{\node[style=maptrans] (x) {$#1$};\draw(0,-1.3)--(x)--(0,1.3);}\,}

\newcommand{\mapconj}[1]{\,\tikz{\node[style=mapconj] (x) {$#1$};\draw(0,-1.3)--(x)--(0,1.3);}\,}

\newcommand{\mapONEtoTWO}[1]{\,\begin{tikzpicture}
  \begin{pgfonlayer}{nodelayer}
    \node [style=map] (0) at (0, 0) {$#1$};
    \node [style=none] (1) at (0, -0.5) {};
    \node [style=none] (2) at (-0.75, 0.5) {};
    \node [style=none] (3) at (-0.75, 1.25) {};
    \node [style=none] (4) at (0, -1.25) {};
    \node [style=none] (5) at (0.75, 0.5) {};
    \node [style=none] (6) at (0.75, 1.25) {};
  \end{pgfonlayer}
  \begin{pgfonlayer}{edgelayer}
    \draw (4.center) to (1.center);
    \draw (2.center) to (3.center);
    \draw (5.center) to (6.center);
  \end{pgfonlayer}
\end{tikzpicture}\,}

\newkeycommand{\pointmap}[style=point][1]{\,\tikz{\node[style=\commandkey{style}] (x) at (0,-0.3) {$#1$};\node [style=none] (3) at (0, 1.0) {};\node [style=none] (3) at (0, -1.0) {};\draw (x)--(0,0.7);}\,}
\newcommand{\graypointmap}[1]{\pointmap[style=gray point]{#1}}

\newcommand{\dpointmap}[1]{\,\tikz{\node[style=point,doubled] (x) at (0,-0.3) {$#1$};\draw[boldedge] (x)--(0,0.7);}\,}
\newcommand{\dcopointmap}[1]{\,\tikz{\node[style=copoint, doubled] (x) at (0,0.3) {$#1$};\draw[boldedge] (x)--(0,-0.7);}\,}

\newcommand{\dpoint}[1]{\,\tikz{\node[style=point,doubled] (x) at (0,-0.3) {$#1$};\draw[boldedge] (x)--(0,0.7);}\,}
\newcommand{\dcopoint}[1]{\,\tikz{\node[style=copoint, doubled] (x) at (0,0.3) {$#1$};\draw[boldedge] (x)--(0,-0.7);}\,}

\newcommand{\kpoint}[1]{\,\tikz{\node[style=kpoint] (x) at (0,-0.05) {$#1$};\draw (x)--(0,0.9);}\,}
\newcommand{\kpointconj}[1]{\,\tikz{\node[style=kpoint conjugate] (x) at (0,-0.05) {$#1$};\draw (x)--(0,0.9);}\,}

\newcommand{\kpointdag}[1]{\,\tikz{\node[style=kpoint adjoint] (x) at (0,0.05) {$#1$};\draw (x)--(0,-0.9);}\,}
\newcommand{\kpointadj}[1]{\kpointdag{#1}}
\newcommand{\kpointtrans}[1]{\,\tikz{\node[style=kpoint transpose] (x) at (0,0.05) {$#1$};\draw (x)--(0,-0.9);}\,}

\newkeycommand{\typedkpoint}[style=kpoint,edgestyle=][2]{%
\,\begin{tikzpicture}
  \begin{pgfonlayer}{nodelayer}
    \node [style=none] (0) at (0, 1) {};
    \node [style=\commandkey{style}] (1) at (0, -0.75) {$#2$};
    \node [style=right label] (2) at (0.25, 0.5) {$#1$};
  \end{pgfonlayer}
  \begin{pgfonlayer}{edgelayer}
    \draw [\commandkey{edgestyle}] (1) to (0.center);
  \end{pgfonlayer}
\end{tikzpicture}\,}

\newkeycommand{\typedkpointdag}[style=kpoint adjoint,edgestyle=][2]{%
\,\begin{tikzpicture}
  \begin{pgfonlayer}{nodelayer}
    \node [style=none] (0) at (0, -1) {};
    \node [style=\commandkey{style}] (1) at (0, 0.75) {$#2$};
    \node [style=right label] (2) at (0.25, -0.5) {$#1$};
  \end{pgfonlayer}
  \begin{pgfonlayer}{edgelayer}
    \draw [\commandkey{edgestyle}] (0.center) to (1);
  \end{pgfonlayer}
\end{tikzpicture}\,}

\newcommand{\typedpoint}[2]{\,\begin{tikzpicture}
  \begin{pgfonlayer}{nodelayer}
    \node [style=none] (0) at (0, 1) {};
    \node [style=point] (1) at (0, -0.5) {$#2$};
    \node [style=right label] (2) at (0.25, 0.5) {$#1$};
  \end{pgfonlayer}
  \begin{pgfonlayer}{edgelayer}
    \draw (1) to (0.center);
  \end{pgfonlayer}
\end{tikzpicture}\,}

\newcommand{\bistate}[1]{\,\begin{tikzpicture}
  \begin{pgfonlayer}{nodelayer}
    \node [style=kpoint, minimum width=1 cm, inner sep=2pt] (0) at (0, -0.25) {$#1$};
    \node [style=none] (1) at (-0.75, 0) {};
    \node [style=none] (2) at (0.75, 0) {};
    \node [style=none] (3) at (-0.75, 0.75) {};
    \node [style=none] (4) at (0.75, 0.75) {};
  \end{pgfonlayer}
  \begin{pgfonlayer}{edgelayer}
    \draw (1.center) to (3.center);
    \draw (2.center) to (4.center);
  \end{pgfonlayer}
\end{tikzpicture}\,}

\newcommand{\bistateadj}[1]{\,\begin{tikzpicture}[yshift=-3mm]
  \begin{pgfonlayer}{nodelayer}
    \node [style=kpointadj, minimum width=1 cm, inner sep=2pt] (0) at (0, 0.5) {$\psi$};
    \node [style=none] (1) at (-0.75, 0.25) {};
    \node [style=none] (2) at (0.75, 0.25) {};
    \node [style=none] (3) at (-0.75, -0.5) {};
    \node [style=none] (4) at (0.75, -0.5) {};
  \end{pgfonlayer}
  \begin{pgfonlayer}{edgelayer}
    \draw (1.center) to (3.center);
    \draw (2.center) to (4.center);
  \end{pgfonlayer}
\end{tikzpicture}\,}

\newcommand{\bistatebraket}[2]{\,\begin{tikzpicture}
  \begin{pgfonlayer}{nodelayer}
    \node [style=kpoint, minimum width=1 cm, inner sep=2pt] (0) at (0, -0.75) {$#1$};
    \node [style=none] (1) at (-0.75, -0.5) {};
    \node [style=none] (2) at (0.75, -0.5) {};
    \node [style=none] (3) at (-0.75, 0.5) {};
    \node [style=none] (4) at (0.75, 0.5) {};
    \node [style=kpointadj, minimum width=1 cm, inner sep=2pt] (5) at (0, 0.75) {$#2$};
  \end{pgfonlayer}
  \begin{pgfonlayer}{edgelayer}
    \draw (1.center) to (3.center);
    \draw (2.center) to (4.center);
  \end{pgfonlayer}
\end{tikzpicture}\,}

\newcommand{\binop}[2]{\,\begin{tikzpicture}
  \begin{pgfonlayer}{nodelayer}
    \node [style=none] (0) at (0.75, -1.25) {};
    \node [style=right label, xshift=1 mm] (1) at (0.75, -1) {$#1$};
    \node [style=none] (2) at (0, 1.25) {};
    \node [style=white dot] (3) at (0, 0) {$#2$};
    \node [style=none] (4) at (-0.75, -1.25) {};
    \node [style=right label] (5) at (-0.5, -1) {$#1$};
    \node [style=right label] (6) at (0.25, 1) {$#1$};
  \end{pgfonlayer}
  \begin{pgfonlayer}{edgelayer}
    \draw [bend right=15, looseness=1.00] (0.center) to (3);
    \draw [bend left=15, looseness=1.00] (4.center) to (3);
    \draw (3) to (2.center);
  \end{pgfonlayer}
\end{tikzpicture}\,}

\newcommand{\dkpoint}[1]{\,\tikz{\node[style=dkpoint] (x) at (0,-0.1) {$#1$};\draw[boldedge] (x)--(0,1);}\,}
\newcommand{\dkpointadj}[1]{\,\tikz{\node[style=dkpointadj] (x) at (0,0.1) {$#1$};\draw[boldedge] (x)--(0,-1);}\,}

\newcommand{\trace}{\,\begin{tikzpicture}
  \begin{pgfonlayer}{nodelayer}
    \node [style=none] (0) at (0, -0.75) {};
    \node [style=upground] (1) at (0, 0.75) {};
  \end{pgfonlayer}
  \begin{pgfonlayer}{edgelayer}
    \draw [style=boldedge] (0.center) to (1);
  \end{pgfonlayer}
\end{tikzpicture}\,}

\newcommand{\namedeq}[1]{%
\ensuremath{\overset{\,(#1)}{\vphantom{X}=}}\xspace}
\newcommand{\scalareq}{%
\ensuremath{\overset{\diamond}{=}}\xspace}

\newkeycommand{\pointketbra}[style1=copoint,style2=point][2]{\,%
\begin{tikzpicture}
  \begin{pgfonlayer}{nodelayer}
    \node [style=none] (0) at (0, -1.5) {};
    \node [style=\commandkey{style1}] (1) at (0, -0.75) {$#1$};
    \node [style=\commandkey{style2}] (2) at (0, 0.75) {$#2$};
    \node [style=none] (3) at (0, 1.5) {};
  \end{pgfonlayer}
  \begin{pgfonlayer}{edgelayer}
    \draw (0.center) to (1);
    \draw (2) to (3.center);
  \end{pgfonlayer}
\end{tikzpicture}\,}

\newkeycommand{\twocopointketbra}[style1=copoint,style2=copoint,style3=point][3]{\,%
\begin{tikzpicture}
  \begin{pgfonlayer}{nodelayer}
    \node [style=none] (0) at (-0.75, -1.5) {};
    \node [style=none] (1) at (0.75, -1.5) {};
    \node [style=\commandkey{style1}] (2) at (-0.75, -0.75) {$#1$};
    \node [style=\commandkey{style2}] (3) at (0.75, -0.75) {$#2$};
    \node [style=\commandkey{style3}] (4) at (0, 0.75) {$#3$};
    \node [style=none] (5) at (0, 1.5) {};
  \end{pgfonlayer}
  \begin{pgfonlayer}{edgelayer}
    \draw (0.center) to (2);
    \draw (1.center) to (3);
    \draw (4) to (5.center);
  \end{pgfonlayer}
\end{tikzpicture}\,}

\newcommand{\graytwocopointketbra}{\twocopointketbra[style1=gray copoint,style2=gray copoint,style3=gray point]}

\newkeycommand{\twopointketbra}[style1=copoint,style2=point,style3=point][3]{\,%
\begin{tikzpicture}
  \begin{pgfonlayer}{nodelayer}
    \node [style=none] (0) at (0, -1.5) {};
    \node [style=none] (1) at (-0.75, 1.5) {};
    \node [style=\commandkey{style1}] (2) at (0, -0.75) {$#1$};
    \node [style=\commandkey{style2}] (3) at (-0.75, 0.75) {$#2$};
    \node [style=\commandkey{style3}] (4) at (0.75, 0.75) {$#3$};
    \node [style=none] (5) at (0.75, 1.5) {};
  \end{pgfonlayer}
  \begin{pgfonlayer}{edgelayer}
    \draw (0.center) to (2);
    \draw (1.center) to (3);
    \draw (4) to (5.center);
  \end{pgfonlayer}
\end{tikzpicture}\,}

\newcommand{\graytwopointketbra}{\twopointketbra[style1=gray copoint,style2=gray point,style3=gray point]}

\newcommand{\idwire}{\,%
\begin{tikzpicture}
  \begin{pgfonlayer}{nodelayer}
    \node [style=none] (0) at (0, -1.5) {};
    \node [style=none] (3) at (0, 1.5) {};
  \end{pgfonlayer}
  \begin{pgfonlayer}{edgelayer}
    \draw (0.center) to (3.center);
  \end{pgfonlayer}
\end{tikzpicture}\,}

\newcommand{\shortidwire}{\,%
\begin{tikzpicture}
  \begin{pgfonlayer}{nodelayer}
    \node [style=none] (0) at (0, -0.8) {};
    \node [style=none] (3) at (0, 0.8) {};
  \end{pgfonlayer}
  \begin{pgfonlayer}{edgelayer}
    \draw (0.center) to (3.center);
  \end{pgfonlayer}
\end{tikzpicture}\,}

\newkeycommand{\pointbraket}[style1=point,style2=copoint][2]{\,%
\begin{tikzpicture}
  \begin{pgfonlayer}{nodelayer}
    \node [style=\commandkey{style1}] (0) at (0, -0.5) {$#1$};
    \node [style=\commandkey{style2}] (1) at (0, 0.5) {$#2$};
  \end{pgfonlayer}
  \begin{pgfonlayer}{edgelayer}
    \draw (0) to (1);
  \end{pgfonlayer}
\end{tikzpicture}\,}

\newcommand{\graypointbraket}[2]{\,%
\begin{tikzpicture}
  \begin{pgfonlayer}{nodelayer}
    \node [style=gray point] (0) at (0, -0.5) {$#1$};
    \node [style=gray copoint] (1) at (0, 0.5) {$#2$};
  \end{pgfonlayer}
  \begin{pgfonlayer}{edgelayer}
    \draw (0) to (1);
  \end{pgfonlayer}
\end{tikzpicture}\,}

\newkeycommand{\kpointbraket}[style1=kpoint,style2=kpoint adjoint][2]{\,%
\begin{tikzpicture}
  \begin{pgfonlayer}{nodelayer}
    \node [style=\commandkey{style1}] (0) at (0, -0.75) {$#1$};
    \node [style=\commandkey{style2}] (1) at (0, 0.75) {$#2$};
  \end{pgfonlayer}
  \begin{pgfonlayer}{edgelayer}
    \draw (0) to (1);
  \end{pgfonlayer}
\end{tikzpicture}\,}

\newcommand{\kpointbraketconj}[2]{\,%
\begin{tikzpicture}
  \begin{pgfonlayer}{nodelayer}
    \node [style=kpoint conjugate] (0) at (0, -0.75) {$#1$};
    \node [style=kpoint transpose] (1) at (0, 0.75) {$#2$};
  \end{pgfonlayer}
  \begin{pgfonlayer}{edgelayer}
    \draw (0) to (1);
  \end{pgfonlayer}
\end{tikzpicture}\,}

\newcommand{\kpointketbra}[2]{\,%
\begin{tikzpicture}
  \begin{pgfonlayer}{nodelayer}
    \node [style=none] (0) at (0, -2) {};
    \node [style=kpointdag] (1) at (0, -1) {$#1$};
    \node [style=kpoint] (2) at (0, 1) {$#2$};
    \node [style=none] (3) at (0, 2) {};
  \end{pgfonlayer}
  \begin{pgfonlayer}{edgelayer}
    \draw (0.center) to (1);
    \draw (2) to (3.center);
  \end{pgfonlayer}
\end{tikzpicture}\,}

\newcommand{\boxpointmap}[2]{\,%
\begin{tikzpicture}
  \begin{pgfonlayer}{nodelayer}
    \node [style=point] (0) at (0, -1.25) {$#1$};
    \node [style=map] (1) at (0, 0) {$#2$};
    \node [style=none] (2) at (0, 1.25) {};
  \end{pgfonlayer}
  \begin{pgfonlayer}{edgelayer}
    \draw (0) to (1);
    \draw (1) to (2.center);
  \end{pgfonlayer}
\end{tikzpicture}%
\,}

\newcommand{\boxtranspointmap}[2]{\,%
\begin{tikzpicture}
  \begin{pgfonlayer}{nodelayer}
    \node [style=point] (0) at (0, -1.25) {$#1$};
    \node [style=maptrans] (1) at (0, 0) {$#2$};
    \node [style=none] (2) at (0, 1.25) {};
  \end{pgfonlayer}
  \begin{pgfonlayer}{edgelayer}
    \draw (0) to (1);
    \draw (1) to (2.center);
  \end{pgfonlayer}
\end{tikzpicture}%
\,}

\newkeycommand{\kpointmap}[style1=kpoint,style2=map][2]{\,%
\begin{tikzpicture}
  \begin{pgfonlayer}{nodelayer}
    \node [style=\commandkey{style1}] (0) at (0, -1.1) {$#1$};
    \node [style=\commandkey{style2}] (1) at (0, 0.2) {$#2$};
    \node [style=none] (2) at (0, 1.5) {};
  \end{pgfonlayer}
  \begin{pgfonlayer}{edgelayer}
    \draw (0) to (1);
    \draw (1) to (2.center);
  \end{pgfonlayer}
\end{tikzpicture}%
\,}

\newcommand{\boxpointmapdag}[2]{\,%
\begin{tikzpicture}
  \begin{pgfonlayer}{nodelayer}
    \node [style=point] (0) at (0, -1.25) {$#1$};
    \node [style=mapdag] (1) at (0, 0) {$#2$};
    \node [style=none] (2) at (0, 1.25) {};
  \end{pgfonlayer}
  \begin{pgfonlayer}{edgelayer}
    \draw (0) to (1);
    \draw (1) to (2.center);
  \end{pgfonlayer}
\end{tikzpicture}%
\,}

\newcommand{\boxcopointmap}[2]{\,%
\begin{tikzpicture}
  \begin{pgfonlayer}{nodelayer}
    \node [style=none] (0) at (0, -1.25) {};
    \node [style=map] (1) at (0, 0) {$#1$};
    \node [style=copoint] (2) at (0, 1.25) {$#2$};
  \end{pgfonlayer}
  \begin{pgfonlayer}{edgelayer}
    \draw (0.center) to (1);
    \draw (1) to (2);
  \end{pgfonlayer}
\end{tikzpicture}%
\,}

\newcommand{\sandwichmap}[3]{\,%
\begin{tikzpicture}
  \begin{pgfonlayer}{nodelayer}
    \node [style=point] (0) at (0, -1.25) {$#1$};
    \node [style=map] (1) at (0, 0) {$#2$};
    \node [style=copoint] (2) at (0, 1.25) {$#3$};
  \end{pgfonlayer}
  \begin{pgfonlayer}{edgelayer}
    \draw (0) to (1);
    \draw (1) to (2);
  \end{pgfonlayer}
\end{tikzpicture}%
}

\newcommand{\kpointsandwichmap}[3]{\,%
\begin{tikzpicture}
  \begin{pgfonlayer}{nodelayer}
    \node [style=kpoint] (0) at (0, -1.5) {$#1$};
    \node [style=map] (1) at (0, 0) {$#2$};
    \node [style=kpointdag] (2) at (0, 1.5) {$#3$};
  \end{pgfonlayer}
  \begin{pgfonlayer}{edgelayer}
    \draw (0) to (1);
    \draw (1) to (2);
  \end{pgfonlayer}
\end{tikzpicture}%
}

\newcommand{\kpointsandwichmapdag}[3]{\,%
\begin{tikzpicture}
  \begin{pgfonlayer}{nodelayer}
    \node [style=kpoint] (0) at (0, -1.5) {$#1$};
    \node [style=mapdag] (1) at (0, 0) {$#2$};
    \node [style=kpointdag] (2) at (0, 1.5) {$#3$};
  \end{pgfonlayer}
  \begin{pgfonlayer}{edgelayer}
    \draw (0) to (1);
    \draw (1) to (2);
  \end{pgfonlayer}
\end{tikzpicture}%
}

\newcommand{\sandwichmapdag}[3]{\,%
\begin{tikzpicture}
  \begin{pgfonlayer}{nodelayer}
    \node [style=point] (0) at (0, -1.25) {$#1$};
    \node [style=mapdag] (1) at (0, 0) {$#2$};
    \node [style=copoint] (2) at (0, 1.25) {$#3$};
  \end{pgfonlayer}
  \begin{pgfonlayer}{edgelayer}
    \draw (0) to (1);
    \draw (1) to (2);
  \end{pgfonlayer}
\end{tikzpicture}%
}

\newcommand{\sandwichmaptrans}[3]{\,%
\begin{tikzpicture}
  \begin{pgfonlayer}{nodelayer}
    \node [style=point] (0) at (0, -1.25) {$#1$};
    \node [style=maptrans] (1) at (0, 0) {$#2$};
    \node [style=copoint] (2) at (0, 1.25) {$#3$};
  \end{pgfonlayer}
  \begin{pgfonlayer}{edgelayer}
    \draw (0) to (1);
    \draw (1) to (2);
  \end{pgfonlayer}
\end{tikzpicture}%
}

\newcommand{\sandwichmapconj}[3]{\,%
\begin{tikzpicture}
  \begin{pgfonlayer}{nodelayer}
    \node [style=point] (0) at (0, -1.25) {$#1$};
    \node [style=mapconj] (1) at (0, 0) {$#2$};
    \node [style=copoint] (2) at (0, 1.25) {$#3$};
  \end{pgfonlayer}
  \begin{pgfonlayer}{edgelayer}
    \draw (0) to (1);
    \draw (1) to (2);
  \end{pgfonlayer}
\end{tikzpicture}%
}

\newkeycommand{\sandwichtwo}[style1=point,style2=map,style3=map,style4=copoint][4]{\,%
\begin{tikzpicture}
  \begin{pgfonlayer}{nodelayer}
    \node [style=\commandkey{style2}] (0) at (0, -0.75) {$#2$};
    \node [style=\commandkey{style3}] (1) at (0, 0.75) {$#3$};
    \node [style=\commandkey{style1}] (2) at (0, -2) {$#1$};
    \node [style=\commandkey{style4}] (3) at (0, 2) {$#4$};
  \end{pgfonlayer}
  \begin{pgfonlayer}{edgelayer}
    \draw (2) to (0);
    \draw (0) to (1);
    \draw (1) to (3);
  \end{pgfonlayer}
\end{tikzpicture}\,}

\newkeycommand{\longbraket}[style1=point,style2=copoint][2]{\,%
\begin{tikzpicture}
  \begin{pgfonlayer}{nodelayer}
    \node [style=\commandkey{style1}] (0) at (0, -2) {$#1$};
    \node [style=\commandkey{style2}] (1) at (0, 2) {$#2$};
  \end{pgfonlayer}
  \begin{pgfonlayer}{edgelayer}
    \draw (0) to (1);
  \end{pgfonlayer}
\end{tikzpicture}\,}

\newkeycommand{\onb}[style=point]{\ensuremath{\left\{\tikz{\node[style=\commandkey{style}] (x) at (0,-0.3) {$j$};\draw (x)--(0,0.7);}\right\}}\xspace}

\newcommand{\whiteonb}{\onb[style=point]}

\newcommand{\grayonb}{\onb[style=gray point]}

\newcommand{\redonb}{\onb[style=red point]}

\newcommand{\greenonb}{\onb[style=green point]}


\newkeycommand{\copointmap}[style=copoint][1]{\,\tikz{\node[style=\commandkey{style}] (x) at (0,0.3) {$#1$};\draw (x)--(0,-0.7);}\,}

\newcommand{\redpointmap}[1]{\,\tikz{\node[style=red point] (x) at (0,-0.3) {$#1$};\draw (x)--(0,0.7);}\,}

\newcommand{\redcopointmap}[1]{\,\tikz{\node[style=red copoint] (x) at (0,0.3) {$#1$};\draw (x)--(0,-0.7);}\,}

\newcommand{\greenpointmap}[1]{\,\tikz{\node[style=green point] (x) at (0,-0.3) {$#1$};\draw (x)--(0,0.7);}\,}

\newcommand{\greencopointmap}[1]{\,\tikz{\node[style=green copoint] (x) at (0,0.3) {$#1$};\draw (x)--(0,-0.7);}\,}

% \newkeycommand{\comult}[style=white dot]{\,\begin{tikzpicture}
%   \begin{pgfonlayer}{nodelayer}
%     \node [style=none] (0) at (0, -1) {};
%     \node [style=none] (1) at (-0.75, 1) {};
%     \node [style=\commandkey{style}] (2) at (0, 0) {};
%     \node [style=none] (3) at (0.75, 1) {};
%   \end{pgfonlayer}
%   \begin{pgfonlayer}{edgelayer}
%     \draw [bend left=15, looseness=1.00] (2) to (1.center);
%     \draw (0.center) to (2);
%     \draw [bend right=15, looseness=1.00] (2) to (3.center);
%   \end{pgfonlayer}
% \end{tikzpicture}\,\xspace}

% \newkeycommand{\mult}[style=white dot]{\,\begin{tikzpicture}
%   \begin{pgfonlayer}{nodelayer}
%     \node [style=none] (0) at (0, 1) {};
%     \node [style=none] (1) at (-0.75, -1) {};
%     \node [style=\commandkey{style}] (2) at (0, 0) {};
%     \node [style=none] (3) at (0.75, -1) {};
%   \end{pgfonlayer}
%   \begin{pgfonlayer}{edgelayer}
%     \draw [bend right=15, looseness=1.00] (2) to (1.center);
%     \draw (0.center) to (2);
%     \draw [bend left=15, looseness=1.00] (2) to (3.center);
%   \end{pgfonlayer}
% \end{tikzpicture}\,\xspace}


\newcommand{\grayphasepoint}[1]{\,\begin{tikzpicture}
  \begin{pgfonlayer}{nodelayer}
    \node [style=grey phase dot] (0) at (0, -0.5) {$#1$};
    \node [style=none] (1) at (0, 1) {};
  \end{pgfonlayer}
  \begin{pgfonlayer}{edgelayer}
    \draw (0) to (1.center);
  \end{pgfonlayer}
\end{tikzpicture}\,}

\newcommand{\graysquarepoint}[1]{\begin{tikzpicture}
    \begin{pgfonlayer}{nodelayer}
        \node [style=gray square point] (0) at (0, -0.5) {};
        \node [style=none] (1) at (0, 0.75) {};
        \node [style=none] (2) at (0.75, -0.5) {$#1$};
    \end{pgfonlayer}
    \begin{pgfonlayer}{edgelayer}
        \draw (0) to (1.center);
    \end{pgfonlayer}
\end{tikzpicture}}

\newcommand{\phase}[2]{\,\begin{tikzpicture}
    \begin{pgfonlayer}{nodelayer}
        \node [style=none] (0) at (0, 1) {};
        \node [style=#1] (2) at (0, -0) {$#2$};
        \node [style=none] (3) at (0, -1) {};
    \end{pgfonlayer}
    \begin{pgfonlayer}{edgelayer}
        \draw (2) to (0.center);
        \draw (3.center) to (2);
    \end{pgfonlayer}
\end{tikzpicture}\,}

\newcommand{\dphase}[2]{\,\begin{tikzpicture}
    \begin{pgfonlayer}{nodelayer}
        \node [style=none] (0) at (0, 1) {};
        \node [style=#1] (2) at (0, -0) {$#2$};
        \node [style=none] (3) at (0, -1) {};
    \end{pgfonlayer}
    \begin{pgfonlayer}{edgelayer}
        \draw [doubled] (2) to (0.center);
        \draw [doubled] (3.center) to (2);
    \end{pgfonlayer}
\end{tikzpicture}\,}

\newcommand{\measure}[1]{\,\begin{tikzpicture}
    \begin{pgfonlayer}{nodelayer}
        \node [style=none] (0) at (0, 0.5) {};
        \node [style=#1] (2) at (0, -0) {};
        \node [style=none] (3) at (0, -0.5) {};
    \end{pgfonlayer}
    \begin{pgfonlayer}{edgelayer}
        \draw (2) to (0.center);
        \draw[doubled] (3.center) to (2);
    \end{pgfonlayer}
\end{tikzpicture}\,}

\newcommand{\prepare}[1]{\,\begin{tikzpicture}
    \begin{pgfonlayer}{nodelayer}
        \node [style=none] (0) at (0, 0.5) {};
        \node [style=#1] (2) at (0, -0) {};
        \node [style=none] (3) at (0, -0.5) {};
    \end{pgfonlayer}
    \begin{pgfonlayer}{edgelayer}
        \draw[doubled] (2) to (0.center);
        \draw (3.center) to (2);
    \end{pgfonlayer}
\end{tikzpicture}\,}

\newcommand{\whitephase}[1]{\phase{white phase dot}{#1}}
\newcommand{\whitedphase}[1]{\dphase{white phase ddot}{#1}}
\newcommand{\greyphase}[1]{\phase{grey phase dot}{#1}}
\newcommand{\greydphase}[1]{\dphase{grey phase ddot}{#1}}

\newcommand{\whitephasepoint}[1]{\,\begin{tikzpicture}
  \begin{pgfonlayer}{nodelayer}
    \node [style=white phase dot] (0) at (0, -0.5) {$#1$};
    \node [style=none] (1) at (0, 1) {};
  \end{pgfonlayer}
  \begin{pgfonlayer}{edgelayer}
    \draw (0) to (1.center);
  \end{pgfonlayer}
\end{tikzpicture}\,}

\newcommand{\pointcopointmap}[2]{\,\tikz{
  \node[style=copoint] (x) at (0,-0.7) {$#2$};\draw (x)--(0,-1.5);
  \node[style=point] (y) at (0,0.7) {$#1$};\draw (0,1.5)--(y);
}\,}

\newcommand{\innerprodmap}[2]{\,\tikz{
\node[style=copoint] (y) at (0,0.5) {$#1$};
\node[style=point] (x) at (0,-0.5) {$#2$};
\draw (x)--(y);}\,}

\newcommand{\wginnerprodmap}[2]{\,\begin{tikzpicture}
    \begin{pgfonlayer}{nodelayer}
        \node [style=point] (0) at (0, -0.75) {$#2$};
        \node [style=gray copoint] (1) at (0, 0.75) {$#1$};
    \end{pgfonlayer}
    \begin{pgfonlayer}{edgelayer}
        \draw (0) to (1);
    \end{pgfonlayer}
\end{tikzpicture}\,}

\def\alpvec{[\vec{\alpha}]}
\def\alpprevec{\vec{\alpha}}
\def\betvec{[\vec{\beta}_j]}
\def\betprevec{\vec{\beta}_j}

\def\blackmu{\mu_{\smallblackdot}} 
\def\graymu {\mu_{\smallgraydot}}
\def\whitemu{\mu_{\smallwhitedot}}

\def\blacketa{\eta_{\smallblackdot}} 
\def\grayeta {\eta_{\smallgraydot}}
\def\whiteeta{\eta_{\smallwhitedot}}

\def\blackdelta{\delta_{\smallblackdot}} 
\def\graydelta {\delta_{\smallgraydot}}
\def\whitedelta{\delta_{\smallwhitedot}}

\def\blackepsilon{\epsilon_{\smallblackdot}} 
\def\grayepsilon {\epsilon_{\smallgraydot}}
\def\whiteepsilon{\epsilon_{\smallwhitedot}}

% \def\whiteeta{\eta_{\!\smallwhitedot}}
% \def\whitevarepsilon{\varepsilon_{\!\smallwhitedot}}

\def\whitePhi{\Phi_{\!\smallwhitedot}}
\def\graymeas{m_{\!\smallgraydot}}

\newcommand{\Owg}{\ensuremath{{\cal O}_{\!\smallwhitedot\!,\!\smallgraydot}}}
\newcommand{\whiteK}{\ensuremath{{\cal  K}_{\!\smallwhitedot}}}
\newcommand{\grayK}{\ensuremath{{\cal  K}_{\!\smallgraydot}}}


% BRAS AND KETS
\newcommand{\bra}[1]{\ensuremath{\left\langle #1 \right|}}
\newcommand{\ket}[1]{\ensuremath{\left|  #1 \right\rangle}}
\newcommand{\roundket}[1]{\ensuremath{\left|  #1 \right)}}
\newcommand{\braket}[2]{\ensuremath{\langle#1|#2\rangle}}
\newcommand{\ketbra}[2]{\ensuremath{\ket{#1}\!\bra{#2}}}

\newcommand{\ketGHZ} {\ket{\textit{GHZ}\,}}
\newcommand{\ketW}   {\ket{\textit{W\,}}}
\newcommand{\ketBell}{\ket{\textit{Bell\,}}}
\newcommand{\ketEPR} {\ket{\textit{EPR\,}}}
\newcommand{\ketGHZD}{\ket{\textrm{GHZ}^{(D)}}}
\newcommand{\ketWD}  {\ket{\textrm{W}^{(D)}}}

\newcommand{\braGHZ} {\bra{\textit{GHZ}\,}}
\newcommand{\braW}   {\bra{\textit{W\,}}}
\newcommand{\braBell}{\bra{\textit{Bell\,}}}
\newcommand{\braEPR} {\bra{\textit{EPR\,}}}


% CATEGORY VARIABLES
\newcommand{\catC}{\ensuremath{\mathcal{C}}\xspace}
\newcommand{\catCop}{\ensuremath{\mathcal{C}^{\mathrm{op}}}\xspace}
\newcommand{\catD}{\ensuremath{\mathcal{D}}\xspace}
\newcommand{\catDop}{\ensuremath{\mathcal{D}^{\mathrm{op}}}\xspace}


% STANDARD CATEGORIES
\newcommand{\catSet}{\ensuremath{\textrm{\bf Set}}\xspace}
\newcommand{\catRel}{\ensuremath{\textrm{\bf Rel}}\xspace}
\newcommand{\catFRel}{\ensuremath{\textrm{\bf FRel}}\xspace}
\newcommand{\catVect}{\ensuremath{\textrm{\bf Vect}}\xspace}
\newcommand{\catFVect}{\ensuremath{\textrm{\bf FVect}}\xspace}
\newcommand{\catFHilb}{\ensuremath{\textrm{\bf FHilb}}\xspace}
\newcommand{\catHilb}{\ensuremath{\textrm{\bf Hilb}}\xspace}
\newcommand{\catSuperHilb}{\ensuremath{\textrm{\bf SuperHilb}}\xspace}
\newcommand{\catAb}{\ensuremath{\textrm{\bf Ab}}\xspace}
\newcommand{\catTop}{\ensuremath{\textrm{\bf Top}}\xspace}
\newcommand{\catCHaus}{\ensuremath{\textrm{\bf CHaus}}\xspace}
\newcommand{\catHaus}{\ensuremath{\textrm{\bf Haus}}\xspace}
\newcommand{\catGraph}{\ensuremath{\textrm{\bf Graph}}\xspace}
\newcommand{\catMat}{\ensuremath{\textrm{\bf Mat}}\xspace}
\newcommand{\catGr}{\ensuremath{\textrm{\bf Gr}}\xspace}
\newcommand{\catSpek}{\ensuremath{\textrm{\bf Spek}}\xspace}




% ========================
% = COMMUTATIVE DIAGRAMS =
% ========================

\tikzstyle{cdiag}=[matrix of math nodes, row sep=3em, column sep=3em, text height=1.5ex, text depth=0.25ex,inner sep=0.5em]
\tikzstyle{arrow above}=[transform canvas={yshift=0.5ex}]
\tikzstyle{arrow below}=[transform canvas={yshift=-0.5ex}]

\newcommand{\csquare}[8]{
\begin{tikzpicture}
    \matrix(m)[cdiag,ampersand replacement=\&]{
    #1 \& #2 \\
    #3 \& #4  \\};
    \path [arrs] (m-1-1) edge node {$#5$} (m-1-2)
                 (m-2-1) edge node {$#6$} (m-2-2)
                 (m-1-1) edge node [swap] {$#7$} (m-2-1)
                 (m-1-2) edge node {$#8$} (m-2-2);
\end{tikzpicture}
}

% commands for putting pushout/pullback brackets on commutative diags
\newcommand{\NWbracket}[1]{%
\draw #1 +(-0.25,0.5) -- +(-0.5,0.5) -- +(-0.5,0.25);}
\newcommand{\NEbracket}[1]{%
\draw #1 +(0.25,0.5) -- +(0.5,0.5) -- +(0.5,0.25);}
\newcommand{\SWbracket}[1]{%
\draw #1 +(-0.25,-0.5) -- +(-0.5,-0.5) -- +(-0.5,-0.25);}
\newcommand{\SEbracket}[1]{%
\draw #1 +(0.25,-0.5) -- +(0.5,-0.5) -- +(0.5,-0.25);}
\newcommand{\THETAbracket}[2]{%
\draw [rotate=#1] #2 +(0.25,0.5) -- +(0.5,0.5) -- +(0.5,0.25);}

\newcommand{\posquare}[8]{
\begin{tikzpicture}
    \matrix(m)[cdiag,ampersand replacement=\&]{
    #1 \& #2 \\
    #3 \& #4  \\};
    \path [arrs] (m-1-1) edge node {$#5$} (m-1-2)
                 (m-2-1) edge node [swap] {$#6$} (m-2-2)
                 (m-1-1) edge node [swap] {$#7$} (m-2-1)
                 (m-1-2) edge node {$#8$} (m-2-2);
    \NWbracket{(m-2-2)}
\end{tikzpicture}
}

\newcommand{\pbsquare}[8]{
\begin{tikzpicture}
    \matrix(m)[cdiag,ampersand replacement=\&]{
    #1 \& #2 \\
    #3 \& #4  \\};
    \path [arrs] (m-1-1) edge node {$#5$} (m-1-2)
                 (m-2-1) edge node [swap] {$#6$} (m-2-2)
                 (m-1-1) edge node [swap] {$#7$} (m-2-1)
                 (m-1-2) edge node {$#8$} (m-2-2);
  \SEbracket{(m-1-1)}
\end{tikzpicture}
}

\newcommand{\ctri}[6]{
    \begin{tikzpicture}[-latex]
        \matrix (m) [cdiag,ampersand replacement=\&] { #1 \& #2 \\ #3 \& \\ };
        \path [arrs] (m-1-1) edge node {$#4$} (m-1-2)
              (m-1-1) edge node [swap] {$#5$} (m-2-1)
              (m-2-1) edge node [swap] {$#6$} (m-1-2);
    \end{tikzpicture}
}

% \newcommand{\crun}[5]{
% \ensuremath{#1 \overset{#2}{\longrightarrow} #3 \overset{#4}{\longrightarrow} #5}
% }

\newcommand{\carr}[3]{
\begin{tikzpicture}
    \matrix(m)[cdiag,ampersand replacement=\&]{
    #1 \& #3 \\};
    \path [arrs] (m-1-1) edge node {$#2$} (m-1-2);
\end{tikzpicture}
}

\newcommand{\crun}[5]{
\begin{tikzpicture}
    \matrix(m)[cdiag,ampersand replacement=\&]{
    #1 \& #3 \& #5 \\};
    \path [arrs] (m-1-1) edge node {$#2$} (m-1-2)
                 (m-1-2) edge node {$#4$} (m-1-3);
\end{tikzpicture}
}

\newcommand{\cspan}[5]{
\ensuremath{#1 \overset{#2}{\longleftarrow} #3
               \overset{#4}{\longrightarrow} #5}
}

\newcommand{\ccospan}[5]{
\ensuremath{#1 \overset{#2}{\longrightarrow} #3
               \overset{#4}{\longleftarrow} #5}
}

\newcommand{\cpair}[4]{
\begin{tikzpicture}
    \matrix(m)[cdiag,ampersand replacement=\&]{
    #1 \& #2 \\};
    \path [arrs] (m-1-1.20) edge node {$#3$} (m-1-2.160)
                 (m-1-1.-20) edge node [swap] {$#4$} (m-1-2.-160);
\end{tikzpicture}
}

\newcommand{\csquareslant}[9]{
\begin{tikzpicture}[-latex]
    \matrix(m)[cdiag,ampersand replacement=\&]{
    #1 \& #2 \\
    #3 \& #4  \\};
    \path [arrs] (m-1-1) edge node {$#5$} (m-1-2)
                 (m-2-1) edge node {$#6$} (m-2-2)
                 (m-1-1) edge node [swap] {$#7$} (m-2-1)
                 (m-1-2) edge node {$#8$} (m-2-2)
                 (m-1-2) edge node [swap] {$#9$} (m-2-1);
\end{tikzpicture}
}


